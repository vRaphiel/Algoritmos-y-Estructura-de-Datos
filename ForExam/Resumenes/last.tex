\documentclass{article}
\input{Algo1Macros}

\begin{document}

\section{Machete Parcial}

\subsection{Especificacion lista de Partes}
Dada una secuencia l con todos sus elementos distintos, devolver la secuencia de partes; es decir, la secuencia de todas las secuencias incluidas en l, cada una con sus elementos en el mismo orden en que aparecen en l.
Características de la entrada:
\begin{itemize}
    \item por simpleza, supongamos una secuencia de enteros "l"
    \item todos sus elementos son distintos (ojo: no necesariamente cumplen un orden)
\end{itemize}
Caracerísticas de la salida
\begin{itemize}
    \item secuencia de secuencias de enteros "res"
    \item $\longitud{res}$ = 2$^{\longitud{l}}$ (Recordando algebra 1)
    \item no hay secuencias repetidas en res
    \item todos los elementos de res son "sublistas" de l 
        (importante: "...cada una con sus elementos en el mismo orden en que aparecen en l.")
\end{itemize}
¿Qué sería una "sublista" de l?
Supongamos que se llama r, entonces:
\begin{itemize}
    \item  todos los elementos de r están en l
    \item  r no tiene elementos repetidos
    \item  los elementos de r se encuentran en el mismo oreden en que aparecen en l
\end{itemize}
Ahora sí, especifiquemoslo.

\begin{proc}{partes}{in l:\TLista{\ent}, out res:\bool}{}
    \pre{ \neg hayRepetidos(l)}
    \post{ |res|=2^{|l|} \wedge \neg hayRepetidos(res) \wedge todasSublistas(res, l)}
    \pred{hayRepetidos}{l:\TLista{\ent}}{\\(\exists i,j:\ent)(0\leq i,j < |l| \wedge i \neq j \wedge todasSublistas(res,l))}
    \pred{todasSublistas}{res:\matriz{\ent}, l:\TLista{\ent}}{\\(\forall r:\TLista{\ent})(r\in res \implica esSublista(r,l))}
    \pred{esSublista}{r:\TLista{\ent}, l\TLista{\ent}}{\\estaContenida(r,l) \wedge \neq hayRepetidos(r) \wedge respetaOrden(r,l)}
    \pred{estaContenida}{r:\TLista{\ent}, l:\TLista{\ent}}{\\(\forall x:\ent)(x\in r \implica x \in l)}
    \pred{respetaOrden}{r:\TLista{\ent}, l:\TLista{\ent}}{\\(\forall i:\ent)(0\leq i < (|r|-1) \implicaLuego estaAntes(r[i], r[i+1], l))}
    \pred{estaAntes}{a,b:\ent, l:\TLista{\ent}}{\\(\exists i,j:\ent)(0\leq i,j<|l| \yLuego s[i]=a \wedge s[j]=b \wedge i<j)}
\end{proc}

\pagebreak

\section{Tiempo de ejecucion en peor caso}

\subsection{Ejercicio 2a}

\begin{lstlisting}
// Si contamos division y resta entonces
void f(vector<int> &v){             // n = |v|
    int i = v.size();               // 2
    while(i >= 0){                  // 1, n/2 ciclos
        v[v.size()/2 - i] = i;      // 3
        v[v.size()/2 + i] = i;      // 3
        i--;                        // 1
    }                               // t(n) = 1 + 4n
}
\end{lstlisting}
\begin{equation}
    t(n) = 3 + 4n = O(n)
\end{equation}

\subsection{Ejercicio 2b}

\begin{lstlisting}
// Pre: e pertenece a v1
int f(vector<int> &v1, int e){  // n = |v|
    int i = 0;                  // 1
    while(v1[i] != e){          // 1 (Distinccion), itera mientras no coincida,
                                // busco el peor caso donde recorro todo el vector
                                // n iteraciones
        i++                     // 1 (Incremento)
    }                           // t(n) = 1 + 2n
    return i;                   
}
\end{lstlisting}
\begin{equation}
    t(n) = 2 + 2n = O(n)
\end{equation}

\subsection{Ejercicio 2c}

\begin{lstlisting}
push_back tiene costo 0(1);

void f(vector<int> &v1, vector<int> &v2){   // n = |v1| m = |v2|
    vector<int> res();                      // 1
    for(int i = 0; i<v1.size(); i++){       // 2 (init, guard), n iteraciones
        res.push_back(v1[i]);               // 1
    }                                       // t(n) = 2 + 2n

    for(int i = 0; i<v2.size(); i++){       // 2, n iteraciones
        res.push_back(v2[i]);               // 1
    }                                       // t(m) = 2 + 2n;
    return res;                             
}
\end{lstlisting}
\begin{equation}
    t(n, m) = 5 + 2n + 2m = O(n + m)
\end{equation}


\subsection{Ejercicio 3}    
Escribir un programa que sea correcto respecto de la especificacion y cuyo 
tiempo de ejecucion de peor caso pertenezca a O(\longitud{s}) donde s es la secuencia pasada
como parametro\\ \\
\begin{proc}{restarLosPares}{in s:\TLista{\ent}, in x:\ent, out res:\TLista{\ent}}{}
    \pre{\True}
    \post{|res| = |s| \yLuego (\forall i:\ent)(0\leq i < |s| \implicaLuego \\
    res[i] = x - \sum_{j=0}^i \IfThenElse{esPar(s[j])}{s[j]}{0})}
\end{proc}

\begin{lstlisting}
vector<int> restarLosPares(vector<int> s, int x){   // O(|s|)
    vector<int> res;                    // 1
    int i = 0;                          // 1
    int suma = 0;                       // 1
    while(i < s.size()){                // 1, |s| iteraciones
        if(s[i] % 2 == 0){              // 2
            suma += S[i];               // 1
        }                               
        res.push_back(x - suma);        // 2
        i++;                            
    }                                   // t(n) = 1 + 6n
    return res;
}
\end{lstlisting}
\begin{equation}
    t(n) = 4 + 6n = O(n) -> n = |s| -> O(|s|)
\end{equation}

\subsection{Ejercicio 4}    
Una matriz cuadrada se dice triangular si todos los elementos por debajo de las
diagonal son iguales a 0
\begin{enumerate}
    \item Escribir un programa que calcule el determinante de una matriz triangular.
    Recordar que el determinante de una matriz triangular es el producto de los elementos de su diagonal
    \item Escribir un programa que determine si una matriz de N + M es o no triangular
    \item Calcular el tiempo de ejecucion de peor caso de los programas
    \begin{enumerate}
        \item en funcion de la cantidad de filas de la matriz
        \item en funcion de la cantidad de elementos de la matriz
    \end{enumerate}
\end{enumerate}

\begin{lstlisting}
//1. detTriangular
// Pre: M es matriz cuadrada y es triangular
// Post res = M[0][0] * M[1][1] * .... * M[n-1][n-1]
int detTriangular(vector<vector<int>> M){ 
                            // 3.1 n = |M|          //3.2 n = |M|*|M|
    int i = 0;              // 1                    // 1
    int res = 1;            // 1                    // 1
    while(i < M.size()){    // 1,  n iteraciones    // 1, sqrt(n) iteraciones
        res = res*M[i][i]   // 2                    // 2
        i++;                //                      
    }                       // t(n) = 1 + 3n        // t(n) = 1 + 3*sqrt(n)
    return res;             // t(n) = 3 + 3n        // t(n) = 3 + 3*sqrt(n)
}                           // O(n)                 // O(sqrt(n))
// 2. esTriangular
\end{lstlisting}
\begin{equation}
    Pre: (\forall i:\ent)(0 \leq i < |M| \implicaLuego \longitud{M[i]} = \longitud{M[0]})
\end{equation}
\begin{equation}
        Post: res = true \Iff (\forall i,j:\ent)(0 \leq i < |M| \wedge 0 \leq j < i \implicaLuego \\M[i][j] = 0)
\end{equation}
\begin{lstlisting}
bool esTriangular(vector<vector<int>> M){
                                            // 3.1 n = |M|          //3.2 n = |M|*|M|
    bool res = true;                        // 1                    // 1
    for(int i = 0; i<M.size(); i++){        // 2, n iteraciones     // 2, sqrt(n) iteraciones
        for(int j = 0; j<i; j++){           // 2, i iteraciones     // 
            res = res && M[i][j] == 0;      // 3                    // 
        }                                   // t(i) = 2 + 5i        // 
        // 3.1 t(n) = 2+sum(i=0,n-1)(2+5i)
    }                                       
        // 3.1 t(n) = 4+(2n-2)+5*sum(i=1,n-1)(i)
        // 3.1 t(n) = 4+(2n-2)+5*(n*(n-1) / 2)
        // 3.1 t(n) = 5 + (2n-2)+5*(n*(n-1) / 2)
        // 3.1 t(n) = 3+2n+5(n^2-n)/2
        // 3.1 O(1) + O(n) + O(n^2) = O(n^2)

        // 3.2 Donde dice n tenemos raiz cuadrada de n, entonces
        // lo unico que necesitamos hacer es llegar a que en la cuenta final
        // analizamos reemplazando n^2 por sqrt(n) entonces tenemos que 
        // 3.2 es O(sqrt^2) -> O(n)
    return res;
}
\end{lstlisting}

\subsection{Ejercicio 5}

\begin{proc}{multiplicar}{in m1:\matriz{\ent}, in m2:\matriz{\ent}, out res:\matriz{\ent}}{}
    \pre{\dots}
    \post{|res| = |m1| \yLuego (\forall i:\ent)(0\leq i < |m1| \implicaLuego (|res[i]| = |m2[0]| \yLuego \\
    (\forall j:\ent)(0 \leq j < |m2[0]| \implicaLuego res[i][j] = \sum_{k=0}^{|m2|-1} m1[i][k]*m2[k][j])))}
\end{proc}

\begin{enumerate}
    \item Completar la precondicion del problema
    \item Escribir un programa que retorne AB
    \item Determinar el tiempo de ejecucion de peor caso de este programa en fuuncion de 
    \begin{enumerate}
        \item la cantidad de filas y columnas de cada matriz
        \item Suponiendo que N = filas(m1) = filas(m2) = columnas(m1) = columnas(m2)
    \end{enumerate}
\end{enumerate}

\begin{lstlisting}
// 1 . Pre = {esMatriz(m1) && esMatriz(m2) &&L |m1[0]| = |m2|}
vector<vector<int>> multiplicar(vector<vector<int>> m1, vector<vector<int>> m2){
                        // n = |m1|, m = |m1[0]| = |m2|, r = |m2|
    vector<vector<int>> res;                        // 1
    for(int i = 0; i < m1.size(); i++){             // 2, n iteraciones
        // fila i de res
        temp = vector<int>();                       // 1
        for(int j = 0; j<m2[0].size(); j++){        // 2, m iteraciones
            // columna j de res
            int v = 0;                              // 1
            for(int k = 0; k < m2.size(); k++)      // 2, r iteraciones
                v += m1[i][k] * m2[k][j];           // 2
            temp.push_back(v);                      // 1
        }
        res.push_back(temp);                        // 1
    }
    return res;
}

// t(r) = 2 + 3r
// t(m, r) = 4 + m*(2+3r)
// t(n, m, r) = 2 + n*(4+m*(2+3r))
// t(n, m, r) = 3 + n*(4+m*(2+3r)) -> 3 + n*(4+2m+3mr) = 3 + 4n + 2mn + 3mnr
// El termino que domina es mnr -> O(mnr)

// Si asumo que m = n = r entonces tengo O(n*n*n) = O(n^3)
\end{lstlisting}

\pagebreak
\end{document}
