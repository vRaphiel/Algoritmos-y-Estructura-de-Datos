\documentclass[a4paper]{article}
\usepackage{amsmath}
\input{Algo1Macros.tex}

\begin{document}

\subsection{Busqueda Lineal}

\begin{proc}{contiene}{in t1 :$\TLista{\ent}$, in x:$\ent$, out result:$\bool$}{}
    \pre{\True}
    \post{result = \True \Iff (\exists i:\ent)(0\leq i <|s| \yLuego s[i] = x)}
\end{proc}

El invatiante del ciclo y la funcion variante:

\begin{equation*}
    I \equiv 0 \leq i \leq \longitud{s} \yLuego (\forall j :\ent)(0 \leq j < i \implicaLuego s[j] \neq x) \ \ \ \ \     fv = \longitud{s} - i
\end{equation*}

En c++ la implementacion es 

\begin{lstlisting}
    bool contiene(vector<int> &s, int x){
        int i = 0;
        while( i < s.size() && s[i] != x ){
            i = i+1;
        }
        return i < s.size();
    }
\end{lstlisting}

El tamaño de cada secuencia delimitada la cantidad de iteraciones y, tambien, si el elemento esta contenido en la misma \
Si buscamos que el tiempo de ejecucion sea el maximo, el elemento no debe estar contenido, eso representa el \textbf{peor caso} 
en tiempo de ejecucion
El tiempo de ejecucion de la busqueda lineal es de $O(n)$

\subsection{Busqueda Binaria}
Suponiento que la secuencia esta \textbf{ordenada}

\begin{proc}{contieneOrdenada}{in t1 :$\TLista{\ent}$, in x:$\ent$, out result:$\bool$}{}
    \pre{ordenado(s)}
    \post{result = \True \Iff (\exists i:\ent)(0\leq i <|s| \yLuego s[i] = x)}
\end{proc}

El invatiante del ciclo y la funcion variante:

\begin{equation*}
    I \equiv 0 \leq i < j < \longitud{s} \yLuego s[i] \leq x < s[j] \ \ \ \ \     fv = j - i - 1
\end{equation*}

Y la funcion variante

Le implementacion en C++ es
\begin{lstlisting}[mathescape=true]
    bool contieneOrdenada(vector<int> &s, int x){
        if(s.size() == 0){
            return false;
        } else if(s.size() == 1){
            return s[0] = x;
        } else if(x < s[0]){
            return false;
        } else if(x $\geq$ s[s.size() - 1]){
            return s[s.size()-1] == x;
        } else {
            int low = 0;
            int hight = s.size() -1;
            while(low + 1 < high){
                int mid = (low+high) / 2;
                if(s[mid] $\leq$ x){
                    low = mid;
                }else{
                    high = mid;
                }
            }
            return s[low] = x;
        }
    }    
\end{lstlisting}

La complejidad de la busqueda binaria es $O(log_{2} \longitud{s})$ o $O(log n)$


\end{document}
